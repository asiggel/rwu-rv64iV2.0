\paragraph{GPIO Peripheral Kernel: } The GPIO kernel provides the specific functionality of the GPIO interface, that is described in Section \ref{subsec:gpiofunc02}. The peripheral kernel has its own kernel clock domain.

\paragraph{Bus Peripheral Interface BPI: } The BPI is responsible for the interconnection of the GPIO registers with an on-chip bus. Also the peripheral clock gating is controlled by the BPI.

\paragraph{asSlaveBPI Block: } The asSlaveBPI block lies within the bus clock domain and contains the following subblocks:
\begin{description}
  \item [Special Function Register Block] The Special Function Register Block contains all FIFO specific registers and all special function registers (SFR) of the peripheral kernel that are accessible by the SW via the system bus.
  \item [Service Request Block SRB] The SRB is used to prepare the interrupt and data transfer requests for the Interrupt Control Unit (ICU) and the DMA Controller (DMAC), respectively.
  \item [TX FIFO] The TX FIFO is used to buffer the transmission data from the bus, in order to adapt the character processing speed of the peripheral kernel to the transfer rate of the system bus.
  \item [RX FIFO] The RX FIFO is used to buffer the received data from the kernel, in order to adapt the character processing speed of the peripheral kernel to the transfer rate of the bus system.
\end{description}

\paragraph{Clock Gating Block: } The Clock Gating Block is used to derive the necessary clocks for the several blocks within the peripheral from the clocks provided by the system.

\paragraph{Synchronization Block: } The Synchronization Block is used to synchronize the signals between the bus clock and kernel clock domains.
